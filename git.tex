\documentclass{article}

% Language setting
% Replace `english' with e.g. `spanish' to change the document language
\usepackage[english]{babel}

% Set page size and margins
% Replace `letterpaper' with `a4paper' for UK/EU standard size
\usepackage[letterpaper,top=2cm,bottom=2cm,left=3cm,right=3cm,marginparwidth=1.75cm]{geometry}

% Useful packages
\usepackage{amsmath}
\usepackage{graphicx}
\usepackage[colorlinks=true, allcolors=blue]{hyperref}

\title{section 1}
\author{Mehdi Vatanzadeh}

\begin{document}
\maketitle

\section{Git and GitHub}

\subsection{Repository Initialization and Commits}
\textbf{Step 1: Creating a GitHub Repository}

Go to GitHub and sign in to your account.\\

Click on the + icon in the top right corner and select "New repository."\\

Fill in the repository details:\\

Repository name: Choose a name like FinalAssignment or something unique.\\

Description: Add a brief description of your project.\\

Visibility: Choose public or private based on your preference.\\

Click "Create repository."
\\
\textbf{Step 2: Cloning the Repository to Your Local Machine}\\

On the repository page, click on the "Code" button and copy the repository URL.\\

Open your terminal or Git Bash on your computer.\\

Navigate to the directory where you want to clone the repository\\

Clone the repository\\

\textbf{Step 3: Initializing and Committing Changes}\\

Navigate into your repository folder\\

Create your LaTeX document, for example, a file called assignment.tex\\

Add the files to staging\\

Commit the changes\\

\textbf{Step 4: Pushing Changes to GitHub}\\

Push your initial commit to the GitHub repository\\


\subsection{GitHub Actions for LaTeX Compilation\\}
Setting Up GitHub Actions:\\
In your repository, navigate to the "Actions" tab.\\

Click "Set up a workflow yourself" or choose a template that fits your needs.\\

Create a .github/workflows directory in your project folder if it doesn't exist.\\

Inside this directory, create a YAML file for your workflow, for example, latex.yml\\

Commit and push the workflow file\\

\section{Exploration Tasks}
\subsection{Vim Advanced Features}

\subsubsection{macros}

Macros allow you to record a sequence of commands and replay them.\\

To record a macro:\\

In normal mode, press q followed by a character to name the macro, like a.\\

Then type the sequence of commands you want to record.\\

Press q again to stop recording.\\

To replay the macro, press @ followed by the macro's name (like a). For example, @a.\\

\subsubsection{Splits and Buffers}
Using splits to view multiple files simultaneously and buffers to quickly switch between files:\\

To open a new window vertically: :vsplit filename or horizontally: :split filename.\\

To move between windows: press ctrl-w followed by w.\\

Buffers allow quick switching between open files:\\

List buffers: :buffers or :ls.\\

Switch to a specific buffer number: :b (buffer number). For example, :b2.\\

\subsubsection{Vim Plugins}
Plugins can significantly enhance Vim's power and capabilities. A popular plugin manager called vim-plug makes installing and managing plugins easy.\\

NerdTree: For navigating directories.\\

Fugitive: For working with Git within Vim.\\

YouCompleteMe: For intelligent auto-completion.\\

\subsection{Memory profiling}
\subsubsection{Memory Leak}
Memory leaks occur when a program allocates memory but fails to release it back to the system after it’s done using it. This causes the memory to be wasted and unavailable for other processes, leading to increased memory usage and potential system slowdowns or crashes.\\

\subsubsection{Memory profilers}
Valgrind helps in memory debugging and leak detection by tracking memory allocations and deallocations. It's especially useful for identifying memory that was allocated but never freed.\\
Valgrind detects memory leaks by:\\

Monitoring Memory: Tracks and reports unfreed memory.\\

Providing Logs: Offers detailed reports showing where leaks occur.\\

Automatic Checks: Automatically checks memory management during program execution.\\

\subsection{GNU/Linux Bash Scripting}
\subsubsection{fzf}

Fuzzy search allows users to find results even with partial phrases or typos. It uses algorithms to match patterns approximately instead of requiring exact matches. This feature is especially helpful when the user is unsure of the exact name or makes a typo. For instance, typing "project report" instead of "project report" would still yield the correct result.\\
ls I fzf: this command allows the user to interactively search and select a file or directory from the list generated by the \texttt{ls} command using fuzzy search

\subsubsection{using fzf to find favorite PDF}

To select a PDF file from the list using \texttt{fzf}, use the following command: fd -e pdf | fzf\\

\subsubsection{Opening the file using Zathura}

To open the selected PDF file using \texttt{Zathura}, use the following command:

\begin{verbatim}
zathura $(fd -e pdf | fzf)
\end{verbatim}
This command will list all PDF files in the current directory and its subdirectories, allow you to select one interactively using \texttt{fzf}, and then open the selected PDF using \texttt{Zathura}.

\section{Git and FOSS}
\subsubsection{README.md}
This repository contains the following:

\begin{itemize}
    \item LaTeX files for the final project, where instructions and the project are written in a structured way.
    \item Git repository setup, including regular commits, tagging, and using GitHub Actions for automatic compilation.
    \item Various scripts and tools used in the project, such as Bash commands and LaTeX files.
\end{itemize}

\subsubsection{issues}
I do in git hub\\

\subsubsection{FOSS contribution}
really i do not know:).

\end{document}